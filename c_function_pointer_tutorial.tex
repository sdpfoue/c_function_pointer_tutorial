\documentclass[11pt,a4paper]{article}
\usepackage{color}
\usepackage{listings}
\usepackage{xeCJK}
\usepackage{hyperref}
\usepackage{fancyhdr}
\usepackage{xunicode}%
\usepackage{xltxtra} 
\usepackage{fontspec}
\usepackage[top=1 in,bottom=1 in,left=1.25in,right=1.25in]{geometry}
\pagestyle{fancy}
%\usepackage[center,pagestyles]{titlesec}
%\titleformat{\section}{\centering\Large\bfseries}{\S\,\thesection}{1em}{}
%\titleformat{\subsection}{\large\bfseries}{\S\,\thesubsection}{1em}{}
\setCJKmainfont[BoldFont=STHeiti,ItalicFont=STKaiti]{SimSun}
%\setCJKmainfont[BoldFont=STHeiti]{SimSun}
%\setromanfont{Kai} % 楷体
\setromanfont{STFangsong} % 楷体
\newfontfamily{\K}{BiauKai}
\setmonofont[Scale=0.8]{Courier New} % 等寬字型
\XeTeXlinebreaklocale "zh"
\XeTeXlinebreakskip = 0pt plus 1pt minus 0.1pt

\definecolor{mygreen}{rgb}{0,0.6,0}
\definecolor{mygray}{rgb}{0.5,0.5,0.5}
\definecolor{mymauve}{rgb}{0.58,0,0.82}

\newcommand{\tab}[1]{\hspace{.2\textwidth}\rlap{#1}}


\setromanfont[Mapping=tex-text, %
Ligatures={Required,Common}, %
ItalicFont={Times Italic}, %
BoldFont={Apple LiGothic Medium}]%
{BiauKai}

\newfontinstance\rmfont{Times} \newcommand{\nc}[1]{{\rmfont #1}}
\usepackage[parfill]{parskip}
\usepackage[parfill]{parskip}

\begin{document}
\fancyhead{}
\renewcommand{\headrulewidth}{0pt}
%\maketitle
\begin{center}
\textbf{\huge{C语言函数指针教程}}\\
作者:Lars Haendel\\
2005年1月,德国波鸿\\
\href{http://www.newty.de/}{http://www.newty.de/}\\
email: 请到网站查看\\
GNU许可
\end{center}

\tableofcontents

\section{函数指针介绍}
函数指针是非常高效有趣和优雅的编程技巧。你可以用它来替代switch/if语句,来实现你的后期绑定和函数回调。但是,由于它的语法的复杂程度,它在很多计算机书籍和文档中不受待见。即使提到了,也都是很浅的一笔带过。因为你不需要为它分配和收回内存,它造成错误的概率要比普通指针低很多。所有你需要做的就是搞清楚它的意义和相关的语法。但是要搞清楚:你是否真的需要一个函数指针来完成任务。实现自己的后期绑定很漂亮,但如果使用C++现有的结构和方法或许会让你的代码可读性更高。后期绑定的应用场景在运行时,如果你调用了一个普通函数,你的代码需要知道调用哪个函数。它用一个包含所有可能被调用的函数的V-Table。每次调用将会消耗一些性能,你可以用函数指针来代替普通函数来解决这个问题,也许不需要\footnote{现代的编译器已经非常出色。用我的Borland,我可以节省。。。}


\subsection{什么是函数指针}
当你在程序中一个叫做\textit{标记}的点调用一个函数 \textit{DoIt()} 的时候,你只需要把函数的调用放到程序源码的\textit{标记}处,之后你每次编译代码到这个位置,这个方法将被执行,一切都很正常。但是如果你不想在编译的时候确定哪个函数应该被调用需要怎么做?你可能希望在程序运行的时候再决定什么函数应该被调用。或者你希望使用回调函数或在一个函数池中选择一个合适的来执行。当然,可以使用\textit{switch}语句来实现后一种效果,在不同的条件分支中调用你想用的函数。但是还有其它的方法可以实现这个目标:使用函数指针!在下面的例子中,我们将实现一个加减乘除的4种数学运算。首先来用一个\textit{switch语法}来解决这个问题,然后再用函数指针的方式实现同样的目标\footnote{这是个非常简单的例子,所以应该不会有人来用函数指针来实际解决这个问题}。

\lstdefinestyle{customc}{
    breaklines=true,
        frame=L,
        xleftmargin=\parindent,
        language=C,
        showstringspaces=false,
        basicstyle=\footnotesize\ttfamily,
        keywordstyle=\color{green},
        %commentstyle=\color{dark},
        identifierstyle=\color{blue},
}
\lstset{escapechar=@,style=customc}

\begin{lstlisting}
//----------------------------------------------
// 1.2 例子:如何取代一个switch-case
// 任务:用+、-、×、/ 实现一个数学运算


// 四种运算符 ... 下面的一个函数将被使用
// 在运行时靠一个switch来选择或使用函数指针
float Plus    (float a, float b) { return a+b; }
float Minus   (float a, float b) { return a-b; }
float Multiply(float a, float b) { return a*b; }
float Divide  (float a, float b) { return a/b; }


// switch的解决方法 - <运算符> 将决定执行哪个操作
void Switch(float a, float b, char opCode)
{
   float result;

   // 执行操作
   switch(opCode)
   {
      case '+' : result = Plus     (a, b); break;
      case '-' : result = Minus    (a, b); break;
      case '*' : result = Multiply (a, b); break;
      case '/' : result = Divide   (a, b); break;
   }

   cout << "Switch: 2+5=" << result << endl;         // display result
}


// 使用函数指针的解决方法 - <pt2Func> 是一个指向接受2个浮点数为参数,返回一个浮点数
// 的函数。这个指针函数将决定哪个操作应该被执行
void Switch_With_Function_Pointer(float a, float b, float (*pt2Func)(float, float))
{
   float result = pt2Func(a, b);    // call using function pointer

   cout << "Switch replaced by function pointer: 2-5=";  // display result
   cout << result << endl;
}


// 执行示例代码
void Replace_A_Switch()
{
   cout << endl << "Executing function 'Replace_A_Switch'" << endl;

   Switch(2, 5, /* '+' specifies function 'Plus' to be executed */ '+');
   Switch_With_Function_Pointer(2, 5, /* pointer to function 'Minus' */ &Minus);
}
\end{lstlisting}

\emph{重要提示:}每个函数指针总是指向特定特征的函数!对于一个函数指针,所有使用它的函数必须有相同类型的参数和返回值!
\section{函数指针的C和C++的语法}
根据这2个语言的语法,有2种类型的函数指针:一种是指向普通的C语言的函数,或指向C++的静态成员函数。另外一种是指向C++的\emph{非静态}成员函数。基本的区别是所有指向非静态成员函数的需要一个隐藏的参数:一个指向某个实例的\textit{this}指针\footnote{原文:this-pointer}。一定要搞清楚,这两种类型的指针是不相容的。
\subsection{定义一个函数指针}
一个函数指针只不过是一个变量,它必须像其它变量一样先声明。在下面的例子里我们声明2个函数指针,分别是\textit{pt2Function, pt2Member} 和\textit{pt2ConstMember}。他们指向函数,这个函数以一个浮点数和2个字符作为参数并返回一个整数。在C++例子中我们假设我们指向的函数是类\textit{TMyClass}中的非静态成员。

\begin{lstlisting}
// 2.1 定义一个函数指针并初始化为NULL
int (*pt2Function)(float, char, char) = NULL;                        // C
int (TMyClass::*pt2Member)(float, char, char) = NULL;                // C++
int (TMyClass::*pt2ConstMember)(float, char, char) const = NULL;     // C++
\end{lstlisting}

\subsection{调用约定}
通常你不需要关心函数的调用约定:如果你没有特别的指定另外的约定,编译器会将\textit{\_\_cdecl}作为默认约定。如果你相了解更多,继续往下读。调用约定会告诉编译器怎样传递参数和怎样产生一个函数的名字。其它约定的例子有\textit{\_\_stdcall, \_\_pascal, \_\_fastcall}。调用约定属于一个函数的特征:\emph{不同调用约定的函数和函数指针互相间是不兼容的!}对于Borland和微软的编译器,你指定一个特别的调用约定在返回类型和函数或函数指针名之间。对于GNU GCC来说,使用\textit{\_\_attribute\_\_}关键字:定义函数时要接关键字\textit{\_\_attribute\_\_}并用双大括号描述。\footnote{如果你想了解更多:告诉我。如果你想知道函数调用在钩子下是如何工作的,你可以看Paul Carter的PC汇编教程的子程序(Subprograms)那个章节。}
\begin{lstlisting}
// 2.2 define the calling convention
void __cdecl DoIt(float a, char b, char c);                             // Borland and Microsoft
void         DoIt(float a, char b, char c)  __attribute__((cdecl));     // GNU GCC
\end{lstlisting}
\subsection{给函数指针分配一个地址}
给函数指针分配一个地址是非常简单的。你只需要选择一个合适的名字并且知道函数或成员函数的名字。对于大部分编译器来说地址符\&是可选的。你需要知道函数的名称和类名,并且你可以访问函数内部。

\begin{lstlisting}
// 2.3 assign an address to the function pointer
//     Note: Although you may ommit the address operator on most compilers
//     you should always use the correct way in order to write portable code.

// C
int DoIt  (float a, char b, char c){ printf("DoIt\n");   return a+b+c; }
int DoMore(float a, char b, char c)const{ printf("DoMore\n"); return a-b+c; }

pt2Function = DoIt;      // short form
pt2Function = &DoMore;   // correct assignment using address operator


// C++
class TMyClass
{
public:
   int DoIt(float a, char b, char c){ cout << "TMyClass::DoIt"<< endl; return a+b+c;};
   int DoMore(float a, char b, char c) const
         { cout << "TMyClass::DoMore" << endl; return a-b+c; };

   /* more of TMyClass */
};

pt2ConstMember = &TMyClass::DoMore; // correct assignment using address operator
pt2Member = &TMyClass::DoIt; // note: <pt2Member> may also legally point to &DoMore
\end{lstlisting}

\subsection{比较函数指针}
你可以像往常一样使用比较符(==, !=)。下面的例子将会查检\textit{pt2Function}和\textit{pt2ConstMember}是否包含\textit{DoIt}和\textit{TMyClass::DoMore}的地址。如果相等会输出一段文字。
\begin{lstlisting}
// 2.4 comparing function pointers

// C
if(pt2Function >0){                           // check if initialized
   if(pt2Function == &DoIt)
      printf("Pointer points to DoIt\n"); }
else
   printf("Pointer not initialized!!\n");


// C++
if(pt2ConstMember == &TMyClass::DoMore)
   cout << "Pointer points to TMyClass::DoMore" << endl;
\end{lstlisting}

\subsection{用函数指针调用函数}
在C语言中可以通过*号和一个函数指针来调用函数。你也可以直接使用函数指针来替代函数的名字。在C++中的2种操作符“.*”,“->*”在类的实例上来调用一个非静态成员方法。如果调用发生在另一个成员函数内部你还需要使用\textit{this}。
\begin{lstlisting}
// 2.5 calling a function using a function pointer
int result1 = pt2Function    (12, ’a’, ’b’);
int result2 = (*pt2Function) (12, ’a’, ’b’);
TMyClass instance1;
int result3 = (instance1.*pt2Member)(12, ’a’, ’b’);
int result4 = (*this.*pt2Member)(12, ’a’, ’b’);
// C short way
// C
// C++
// C++ if this-pointer can be used
TMyClass* instance2 = new TMyClass;
int result4 = (instance2->*pt2Member)(12, ’a’, ’b’);  // C++, instance2 is a pointer
delete instance2;
\end{lstlisting}

\subsection{如果把函数指针作为一个函数参数}
你可以把函数指针作为一个函数参数来使用。比如,你可能需要把函数指针当作回调函数来使用。下面代码将示例如何把函数指针传入另一个函数,并将一个float和2个char作为参数,一个int作为返回值:
\begin{lstlisting}
//------------------------------------------------------------------------------------
// 2.6 How to Pass a Function Pointer
// <pt2Func> is a pointer to a function which returns an int and takes a float and two char
void PassPtr(int (*pt2Func)(float, char, char))
{
       int result = (*pt2Func)(12, ’a’, ’b’);     // call using function pointer
          cout << result << endl;
}
// execute example code - ’DoIt’ is a suitable function like defined above in 2.1-4
void Pass_A_Function_Pointer()
{
       cout << endl << "Executing ’Pass_A_Function_Pointer’" << endl;
          PassPtr(&DoIt);
}
\end{lstlisting}
\subsection{怎样返回一个函数指针}
函数指针同样可以作为函数的返回值。下面的例子有2种方式演示如何返回函数指针。第一个返回的函数指针需要2个float参数,并返回一个float。如果你想返回一个指向成员函数的指针,你需要修改所有函数指针的定义声明。
\begin{lstlisting}
//------------------------------------------------------------------------------------
// 2.7 How to Return a Function Pointer
//     ’Plus’ and ’Minus’ are defined above. They return a float and take two float
// Direct solution: Function takes a char and returns a pointer to a
// function which is taking two floats and returns a float. <opCode>
// specifies which function to return
float (*GetPtr1(const char opCode))(float, float){
   if(opCode == ’+’)
      return &Plus;
   else
      return &Minus;} // default if invalid operator was passed
// Solution using a typedef: Define a pointer to a function which is taking
// two floats and returns a float
typedef float(*pt2Func)(float, float);
// Function takes a char and returns a function pointer which is defined
// with the typedef above. <opCode> specifies which function to return
pt2Func GetPtr2(const char opCode)
{
   if(opCode == ’+’)
      return &Plus;
   else
      return &Minus; // default if invalid operator was passed
}
// Execute example code
void Return_A_Function_Pointer()
{
   cout << endl << "Executing ’Return_A_Function_Pointer’" << endl;
   // define a function pointer and initialize it to NULL
   float (*pt2Function)(float, float) = NULL;
   pt2Function=GetPtr1(’+’);   // get function pointer from function ’GetPtr1’
   cout << (*pt2Function)(2, 4) << endl;   // call function using the pointer
   pt2Function=GetPtr2(’-’);   // get function pointer from function ’GetPtr2’
   cout << (*pt2Function)(2, 4) << endl;   // call function using the pointer
}
\end{lstlisting}
\subsection{怎样使用一个数组的函数指针}
操作一个数组的函数指针非常有意思。它使你可以使用数组的索引来使用函数。语法看上去比较复杂,经常会引起混乱。下面你将看到2种方式来定义和使用一个数组的函数,分别由C和C++来实现。第一种方式使用\textit{typedef}来实现,第二种方式直接定义数组。你可以根据自己的喜好来选择实现方式。
\begin{lstlisting}
//------------------------------------------------------------------------------------
// 2.8 How to Use Arrays of Function Pointers
// C ---------------------------------------------------------------------------------
// type-definition: ’pt2Function’ now can be used as type
typedef int (*pt2Function)(float, char, char);
// illustrate how to work with an array of function pointers
void Array_Of_Function_Pointers()
{
   printf("\nExecuting ’Array_Of_Function_Pointers’\n");
   // define arrays and ini each element to NULL, <funcArr1> and <funcArr2> are arrays
   // with 10 pointers to functions which return an int and take a float and two char
   // first way using the typedef
   pt2Function funcArr1[10] = {NULL};
   // 2nd way directly defining the array
   int (*funcArr2[10])(float, char, char) = {NULL};
   // assign the function’s address - ’DoIt’ and ’DoMore’ are suitable functions
   // like defined above in 2.1-4
   funcArr1[0] = funcArr2[1] = &DoIt;
   funcArr1[1] = funcArr2[0] = &DoMore;
   /* more assignments */
   // calling a function using an index to address the function pointer
   printf("%d\n", funcArr1[1](12, ’a’, ’b’));         //  short form
   printf("%d\n", (*funcArr1[0])(12, ’a’, ’b’));      // "correct" way of calling
   printf("%d\n", (*funcArr2[1])(56, ’a’, ’b’));
   printf("%d\n", (*funcArr2[0])(34, ’a’, ’b’));
}
// C++ -------------------------------------------------------------------------------
// type-definition: ’pt2Member’ now can be used as type
typedef int (TMyClass::*pt2Member)(float, char, char);
// illustrate how to work with an array of member function pointers
void Array_Of_Member_Function_Pointers()
{
   cout << endl << "Executing ’Array_Of_Member_Function_Pointers’" << endl;
   // define arrays and ini each element to NULL, <funcArr1> and <funcArr2> are
   // arrays with 10 pointers to member functions which return an int and take
   // a float and two char
   // first way using the typedef
   pt2Member funcArr1[10] = {NULL};
   // 2nd way of directly defining the array
   int (TMyClass::*funcArr2[10])(float, char, char) = {NULL};
   // assign the function’s address - ’DoIt’ and ’DoMore’ are suitable member
   //  functions of class TMyClass like defined above in 2.1-4
   funcArr1[0] = funcArr2[1] = &TMyClass::DoIt;
   funcArr1[1] = funcArr2[0] = &TMyClass::DoMore;
   /* more assignments */
   // calling a function using an index to address the member function pointer
   // note: an instance of TMyClass is needed to call the member functions
   TMyClass instance;
   cout << (instance.*funcArr1[1])(12, ’a’, ’b’) << endl;
   cout << (instance.*funcArr1[0])(12, ’a’, ’b’) << endl;
   cout << (instance.*funcArr2[1])(34, ’a’, ’b’) << endl;
   cout << (instance.*funcArr2[0])(89, ’a’, ’b’) << endl;
}
\end{lstlisting}





\end{document}
