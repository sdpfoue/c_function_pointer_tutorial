\documentclass[11pt,a4paper]{article}
\usepackage{xeCJK}
\usepackage{hyperref}
\usepackage{fancyhdr}
\usepackage{xunicode}%
\usepackage{xltxtra} 
\usepackage{fontspec}
\usepackage[top=1 in,bottom=1 in,left=1.25in,right=1.25in]{geometry}
\pagestyle{fancy}
%\usepackage[center,pagestyles]{titlesec}
%\titleformat{\section}{\centering\Large\bfseries}{\S\,\thesection}{1em}{}
%\titleformat{\subsection}{\large\bfseries}{\S\,\thesubsection}{1em}{}
\setCJKmainfont[BoldFont=STHeiti]{SimSun}
%\setromanfont{Kai} % 楷体
\setromanfont{STFangsong} % 楷体
\newfontfamily{\K}{BiauKai}
\setmonofont[Scale=0.8]{Courier New} % 等寬字型
\XeTeXlinebreaklocale "zh"
\XeTeXlinebreakskip = 0pt plus 1pt minus 0.1pt

\newcommand{\tab}[1]{\hspace{.2\textwidth}\rlap{#1}}


\setromanfont[Mapping=tex-text, %
Ligatures={Required,Common}, %
ItalicFont={Times Italic}, %
BoldFont={Apple LiGothic Medium}]%
{BiauKai}

\newfontinstance\rmfont{Times} \newcommand{\nc}[1]{{\rmfont #1}}
\usepackage[parfill]{parskip}
\usepackage[parfill]{parskip}

\begin{document}
\fancyhead{}
\renewcommand{\headrulewidth}{0pt}
%\maketitle
\begin{center}
\textbf{\huge{C语言函数指针教程}}\\
作者:Lars Haendel\\
2005年1月,德国波鸿\\
\href{http://www.newty.de/}{http://www.newty.de/}\\
email: 请到网站查看\\
GNU许可
\end{center}

\tableofcontents

\section{什么是函数指针}
函数指针是非常高效有趣和优雅的编程技巧。你可以用它来替代switch/if语句,来实现你的后期绑定和函数回调。但是,由于它的语法的复杂程度,它在很多计算机书籍和文档中不受待见。即使提到了,也都是很浅的一笔带过。因为你不需要为它分配和收回内存,它造成错误的概率要比普通指针低很多。所有你需要做的就是搞清楚它的意义和相关的语法。但是要搞清楚:你是否真的需要一个函数指针来完成任务。实现自己的后期绑定很漂亮,但如果使用C++现有的结构和方法或许会让你的代码可读性更高。后期绑定的应用场景在运行时,如果你调用了一个普通函数,你的代码需要知道调用哪个函数。它用一个包含所有可能被调用的函数的V-Table。每次调用将会消耗一些性能,你可以用函数指针来代替普通函数来解决这个问题,也许不需要\footnote{现代的编译器已经非常出色。用我的Borland,我可以节省。。。}
\end{document}
